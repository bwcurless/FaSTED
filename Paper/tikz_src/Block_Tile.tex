\documentclass[tikz,border=2mm]{standalone}

% Binary value zero padding taken from SO
\ExplSyntaxOn
\int_new:N \g_fleet_number_of_zeros
\NewDocumentCommand\SetZeros{m}
{
	\int_gset:Nn \g_fleet_number_of_zeros {#1}
}
\NewDocumentCommand\PrependZeros{om}
{
	\IfValueTF{#1}
	{ \__fleet_count:ne {#1} {#2} }
	{ \__fleet_count:ne {\g_fleet_number_of_zeros} {#2} }
}
\cs_new:Npn \__fleet_count:nn #1#2
{
	\exp_args:Nf \__fleet_prepend:nn
	{ \int_max:nn { #1 - \str_count:n {#2} } { 0 } }
	{#2}
}
\cs_generate_variant:Nn \__fleet_count:nn { ne }
\cs_new:Npn \__fleet_prepend:nn #1#2
{ \prg_replicate:nn {#1}{0} #2 }
\ExplSyntaxOff

% XOR routine taken from SO...
\ExplSyntaxOn
\NewExpandableDocumentCommand{\bitwiseXor}{mm}
{
	\recuenco_bitwise_xor:nn { #1 } { #2 }
}

\cs_new:Nn \recuenco_bitwise_xor:nn
{
	\int_from_bin:e
	{
		\__recuenco_bitwise_xor:ee { \int_to_bin:n { #1 } } { \int_to_bin:n { #2 } }
	}
}
\cs_generate_variant:Nn \int_from_bin:n { e }

\cs_new:Nn \__recuenco_bitwise_xor:nn
{
	\__recuenco_bitwise_xor_binary:ee
	{
		\prg_replicate:nn
		{
			\int_max:nn { \tl_count:n { #1 } } { \tl_count:n { #2 } } - \tl_count:n { #1 }
		}
		{ 0 }
		#1
	}
	{
		\prg_replicate:nn
		{
			\int_max:nn { \tl_count:n { #1 } } { \tl_count:n { #2 } } - \tl_count:n { #2 }
		}
		{ 0 }
		#2
	}
}
\cs_generate_variant:Nn \__recuenco_bitwise_xor:nn { ee }

\cs_new:Nn \__recuenco_bitwise_xor_binary:nn
{
	\__recuenco_bitwise_xor_binary:w #1;#2;
}
\cs_generate_variant:Nn \__recuenco_bitwise_xor_binary:nn { ee }

\cs_new:Npn \__recuenco_bitwise_xor_binary:w #1#2;#3#4;
{
	\int_abs:n { #1-#3 }
	\tl_if_empty:nF { #2 } { \__recuenco_bitwise_xor_binary:w #2;#4; }
}

\ExplSyntaxOff


\definecolor{colColor0}{RGB}{255, 102, 102} % Soft Red  
\definecolor{colColor1}{RGB}{255, 140, 115} % Coral  
\definecolor{colColor2}{RGB}{255, 178, 140} % Soft Orange  
\definecolor{colColor3}{RGB}{230, 153, 190} % Warm Pink  
\definecolor{colColor4}{RGB}{200, 130, 220} % Light Orchid  
\definecolor{colColor5}{RGB}{170, 110, 230} % Soft Violet  
\definecolor{colColor6}{RGB}{140, 90, 240}  % Muted Purple  
\definecolor{colColor7}{RGB}{120, 70, 250}  % Gentle Blue-Purple  
\definecolor{lastColColor}{RGB}{255, 255, 255}  % White
\definecolor{AFragmentColor}{RGB}{255, 0, 0}  % Red
\definecolor{BFragmentColor}{RGB}{0, 0, 255}  % Blue
\definecolor{DFragmentColor}{RGB}{0, 203, 74}  % Green
\definecolor{ASharedMemColor}{RGB}{200, 0, 0}  % Red
\definecolor{BSharedMemColor}{RGB}{0, 0, 200}  % Blue
\definecolor{WarpTileColor}{RGB}{0, 150, 74}  % Green
\definecolor{BlockTileColor}{RGB}{0, 0, 0}  % Black

\definecolor{blockTileColor}{RGB}{0, 255, 0}  % Light Green

\newcommand{\getcolor}[1]{colColor#1}


\def\blockHeight{1}
\def\blockWidth{2}
\pgfmathsetmacro{\halfBlockHeight}{\blockHeight / 2}
\pgfmathsetmacro{\halfBlockWidth}{\blockWidth / 2}

\def\dimColors{colColor0, colColor1, colColor2, colColor3, colColor4, colColor5, colColor6, colColor7, lastColColor}

% Fragment drawings common parameters

% Base rectangle background
\newcommand\whiteSpace{0.5}

	% Compute width of a single register based on the fraction of a single chunk it takes up.
\pgfmathsetmacro\registerWidth{0.75}
\pgfmathsetmacro\registerHeight{0.375}
\pgfmathsetmacro\halfRegisterWidth{\registerWidth / 2}
\pgfmathsetmacro\halfRegisterHeight{\registerHeight / 2}

\newcommand\phasesPercent{0.8}
\pgfmathsetmacro\phaseWidth{\registerWidth * 4}
\pgfmathsetmacro\phaseHeight{\registerHeight * 8}

\pgfmathsetmacro\fragmentWidth{2 * \phaseWidth + (3 * \whiteSpace)}
\pgfmathsetmacro\fragmentHeight{2 * \phaseHeight + (3 * \whiteSpace)}

\tikzstyle fragment=[ultra thick, rounded corners= 5pt]

\newcommand\drawFragment[1]{
	\draw[style=fragment, draw=AFragmentColor] (0, 0) rectangle +(\fragmentWidth, -\fragmentHeight);

	\begin{scope}[xshift=\whiteSpace cm, yshift=-\whiteSpace cm]

	% Draw the four phases
		\foreach \row in {0,  1} {
			\foreach \col in {0,  1} {
				\pgfmathsetmacro\xoffset{\col * (\phaseWidth + \whiteSpace)}
				\pgfmathsetmacro\yoffset{-\row * (\phaseHeight + \whiteSpace)}
				\begin{scope}[xshift=\xoffset cm, yshift=\yoffset cm]
					\pgfmathsetmacro\phaseNum{int(\col * 2 + \row + 1)}
					\node[anchor=south] at (\registerWidth * 2, 0) {Phase \phaseNum};
					% Invoke the Phase drawing section now that we are in our local scope
					#1{\phaseNum}
				\end{scope}
			}
		}
	\end{scope}
}

\newcommand\getPhaseColor[1]{%
	\ifcase #1 \or colColor0%
		\or white%
		\or colColor1%
		\or white%
	\fi
}


\begin{document}

% Block Tile
\begin{tikzpicture}[show background rectangle, 
    background rectangle/.style={style=fragment, fill=BlockTileColor}]

	\newcommand\drawBlockLevelFragment[6]{%
		\def\width{#1}
		\def\height{#2}
		\def\upperLeftContent{#3}
		\def\bottomRightContent{#4}
		\def\color{#5}
		\def\label{#6}

            \def\labelShift{0.075}
		\filldraw[style=fragment, fill= \color, draw=black] (0, 0) node[anchor=south east, font=\tiny, yshift=-\labelShift cm] {\bottomRightContent} rectangle (-\width, \height) node[anchor=north west, font=\tiny, yshift=\labelShift cm] {\upperLeftContent} node[midway, align=center, font=\scriptsize] {\label};
	}


	% Draw the 4 warp tiles in center
	\newcommand\warpTileSize{1.25}
	\newcommand\warpPadding{0.25}
	\pgfmathsetmacro\paddedWarpTileSize{\warpTileSize + \warpPadding}


	% Draw all Warp Tiles
	\begin{scope}[xshift=0 cm, yshift=0 cm]
		\foreach \row in {0, ..., 1} {
			\foreach \col in {0, ..., 1} {
				\begin{scope}[xshift=\paddedWarpTileSize * \col cm, yshift= -\paddedWarpTileSize * \row cm]
					\pgfmathtruncatemacro\warpIndex{\row * 2 + \col + 1}
					\pgfmathtruncatemacro\warpSize{32}

					\pgfmathtruncatemacro\minQueryPoint{\row * \warpSize + 1}
					\pgfmathtruncatemacro\maxQueryPoint{\minQueryPoint + \warpSize - 1}
					\pgfmathtruncatemacro\minCandPoint{\col * \warpSize + 1}
					\pgfmathtruncatemacro\maxCandPoint{\minCandPoint + \warpSize - 1}
					\def\minDim{1}
					\def\maxDim{64}

					\drawBlockLevelFragment{\warpTileSize}{\warpTileSize}{$a_{\minQueryPoint,\minCandPoint}$}{$a_{\maxQueryPoint,\maxCandPoint}$}{WarpTileColor}{Warp Tile\\ \warpIndex}
				\end{scope}
			}
		}
	\end{scope}

	\def\sliceYShift{0.3}
	\def\sliceXShift{0.1}
	\pgfmathtruncatemacro\numSlices{2 }
	\pgfmathtruncatemacro\numSlicesIt{\numSlices - 1}
	% Draw the 64D shared memory chunks that have been paged in
	% A Data
	\begin{scope}[xshift=-\paddedWarpTileSize cm, yshift=-\paddedWarpTileSize cm]
		\foreach \i in {0, ..., \numSlicesIt} {
			\begin{scope}[xshift=-\sliceXShift * \i cm, yshift=\sliceYShift * \i cm]
				\pgfmathtruncatemacro\maxDim{64 * \numSlices - 64 * \i}
				\drawBlockLevelFragment{\warpTileSize * 0.75}{2 * \warpTileSize + \warpPadding}{$p_{1,1}$}{$p_{64,\maxDim}$}{ASharedMemColor}{$P_{\blockf}$\\ (SMEM)}
			\end{scope}
		}
	\end{scope}

	% B Data
	\begin{scope}[xshift=\paddedWarpTileSize cm, yshift=\paddedWarpTileSize cm]
		\pgfmathsetmacro\blockWidth{2 * \warpTileSize + \warpPadding}
		\foreach \i in {0, ..., \numSlicesIt} {
            \begin{scope}[xshift=-\sliceXShift * \i cm, yshift=\sliceYShift * \i cm]
            \pgfmathtruncatemacro\maxDim{64 * \numSlices - 64 * \i}
            \ifnum \i=\numSlicesIt
            \drawBlockLevelFragment{\blockWidth}{\warpTileSize * 0.75}{$p_{1,1}$}{$p_{64,\maxDim}$}{BSharedMemColor}{$Q_{\blockf}$ (SMEM)}
            \else
            % Don't draw center text on earlier stacks
            \drawBlockLevelFragment{\blockWidth}{\warpTileSize * 0.75}{$p_{1,1}$}{$p_{64,\maxDim}$}{BSharedMemColor}{}
            \fi
            \end{scope}
        }
	\end{scope}

  % Draw the entire block tile
	\drawLabeledBoundingBox{color=BlockTileColor}{Single Block Tile}{black}

\end{tikzpicture}
\end{document}
