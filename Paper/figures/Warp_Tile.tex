% Warp Tile
\begin{tikzpicture}
	\newcommand\drawStackableFragment[5]{%
		\def\width{#1}
		\def\height{#2}
		\def\upperLeftContent{#3}
		\def\bottomRightContent{#4}
		\def\color{#5}

		\filldraw[style=fragment, fill= white, draw=\color] (0, 0) node[anchor=south east, font=\tiny, xshift=0.5 mm, yshift=-0.5 mm] {\bottomRightContent} rectangle (-\width, \height) node[anchor=north west, font=\tiny] {\upperLeftContent};
	}


	\newcommand\fullFragWidth{1.5}
	\newcommand\fullFragHeight{1.5}
	\newcommand\halfFragWidth{0.75}
	\newcommand\halfFragPoints{8}
	\newcommand\fullFragPoints{16}
	% All fragments have 16 dimensions
	\newcommand\fragmentDims{16}

	% Can be used to draw stacks of fragments for A or B matrices
	\newcommand\drawFragmentStack[5]{%
		\def\fragColor{#1}
		\def\fragWidth{#2}
		\def\fragWidthInPoints{#3}
		\def\fragIndex{#4}
		\def\fragName{#5}

		% Draw each fragment as a stack of 4
		\newcommand\stacks{4}
		\newcommand\stackOffset{0.4}
		\pgfmathsetmacro\maxStackIndex{int(\stacks - 1)}

		\foreach \i in {0, ..., \maxStackIndex} {
			% Stack up and to the left
			\begin{scope}[xshift=-\i * \stackOffset * 0.125 cm, yshift=\i * \stackOffset cm]
				\pgfmathtruncatemacro\minPoint{\fragWidthInPoints * \fragIndex + 1}
				\pgfmathtruncatemacro\maxPoint{\minPoint + \fragWidthInPoints - 1}

																% Kind of a hack, but we stack upwards so later fragments cover earlier, have
																% to count backwards for dimensions for this to work out cleanly
				\pgfmathtruncatemacro\maxDim{64 - (\i * \fragmentDims)}
				\pgfmathtruncatemacro\minDim{\maxDim - \fragmentDims + 1}

				\drawStackableFragment{\fragWidth}{\fullFragHeight}{$\fragName_{\minPoint,\minDim}$}{$\fragName_{\maxPoint,\maxDim}$}{\fragColor}
			\end{scope}
		}
	}

% Draw the entire warp tile for future reference
	\draw[color=WarpTileColor, style=fragment, xshift=-2 cm, yshift=5 cm] (0,0) rectangle (8.5, -12) node[anchor=south east, color=black] {Single Warp Tile Iteration};

	\newcommand\fragXShift{1}
	\newcommand\fragYShift{3}

	% Draw all D Fragments
	\begin{scope}[xshift=1 cm, yshift=0 cm]
		\foreach \row in {0, ..., 2} {
			\foreach \col in {0, ..., 5} {
				\begin{scope}[xshift=\fragXShift * \col cm, yshift= -\fragYShift * \row cm]
					\pgfmathtruncatemacro\firstQueryPoint{\row * \fullFragPoints + 1}
					\pgfmathtruncatemacro\lastQueryPoint{\firstQueryPoint + \fullFragPoints - 1}
					\pgfmathtruncatemacro\firstCandPoint{\col * \halfFragPoints + 1}
					\pgfmathtruncatemacro\lastCandPoint{\firstCandPoint + \halfFragPoints - 1}


					\drawStackableFragment{\halfFragWidth}{\fullFragHeight}{$D_{\firstQueryPoint,\firstCandPoint}$}{$D_{\lastQueryPoint,\lastCandPoint}$}{DFragmentColor}
				\end{scope}
			}
		}
	\end{scope}

	% Draw all 3 A Fragments
	\begin{scope}[xshift=0 cm, yshift=0 cm]
		\foreach \fragment in {0, ..., 2} {
			\begin{scope}[yshift=-\fragYShift * \fragment cm]
				\drawFragmentStack{AFragmentColor}{\fullFragWidth}{16}{\fragment}{Q}
			\end{scope}
		}
	\end{scope}

	% Draw all 6 B Fragments
	\begin{scope}[xshift=1 cm, yshift=2 cm]
		\foreach \fragment in {0, ..., 5} {
			\begin{scope}[xshift=\fragXShift * \fragment cm]
				\drawFragmentStack{BFragmentColor}{\halfFragWidth}{8}{\fragment}{C}
			\end{scope}
		}

	\end{scope}


\end{tikzpicture}
