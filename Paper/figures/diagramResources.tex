% Binary value zero padding taken from SO
\ExplSyntaxOn
\int_new:N \g_fleet_number_of_zeros
\NewDocumentCommand\SetZeros{m}
{
	\int_gset:Nn \g_fleet_number_of_zeros {#1}
}
\NewDocumentCommand\PrependZeros{om}
{
	\IfValueTF{#1}
	{ \__fleet_count:ne {#1} {#2} }
	{ \__fleet_count:ne {\g_fleet_number_of_zeros} {#2} }
}
\cs_new:Npn \__fleet_count:nn #1#2
{
	\exp_args:Nf \__fleet_prepend:nn
	{ \int_max:nn { #1 - \str_count:n {#2} } { 0 } }
	{#2}
}
\cs_generate_variant:Nn \__fleet_count:nn { ne }
\cs_new:Npn \__fleet_prepend:nn #1#2
{ \prg_replicate:nn {#1}{0} #2 }
\ExplSyntaxOff

% XOR routine taken from SO...
\ExplSyntaxOn
\NewExpandableDocumentCommand{\bitwiseXor}{mm}
{
	\recuenco_bitwise_xor:nn { #1 } { #2 }
}

\cs_new:Nn \recuenco_bitwise_xor:nn
{
	\int_from_bin:e
	{
		\__recuenco_bitwise_xor:ee { \int_to_bin:n { #1 } } { \int_to_bin:n { #2 } }
	}
}
\cs_generate_variant:Nn \int_from_bin:n { e }

\cs_new:Nn \__recuenco_bitwise_xor:nn
{
	\__recuenco_bitwise_xor_binary:ee
	{
		\prg_replicate:nn
		{
			\int_max:nn { \tl_count:n { #1 } } { \tl_count:n { #2 } } - \tl_count:n { #1 }
		}
		{ 0 }
		#1
	}
	{
		\prg_replicate:nn
		{
			\int_max:nn { \tl_count:n { #1 } } { \tl_count:n { #2 } } - \tl_count:n { #2 }
		}
		{ 0 }
		#2
	}
}
\cs_generate_variant:Nn \__recuenco_bitwise_xor:nn { ee }

\cs_new:Nn \__recuenco_bitwise_xor_binary:nn
{
	\__recuenco_bitwise_xor_binary:w #1;#2;
}
\cs_generate_variant:Nn \__recuenco_bitwise_xor_binary:nn { ee }

\cs_new:Npn \__recuenco_bitwise_xor_binary:w #1#2;#3#4;
{
	\int_abs:n { #1-#3 }
	\tl_if_empty:nF { #2 } { \__recuenco_bitwise_xor_binary:w #2;#4; }
}

\ExplSyntaxOff

% New color blind friendly green/blue pallette
\definecolor{colColor0}{HTML}{f7fcf0}
\definecolor{colColor1}{HTML}{e0f3db}
\definecolor{colColor2}{HTML}{ccebc5}
\definecolor{colColor3}{HTML}{a8ddb5}
\definecolor{colColor4}{HTML}{7bccc4}
\definecolor{colColor5}{HTML}{4eb3d3}
\definecolor{colColor6}{HTML}{2b8cbe}
\definecolor{colColor7}{HTML}{0868ac}
%Original Red/orange/purple pallette
%\definecolor{colColor0}{RGB}{255, 102, 102} % Soft Red  
%\definecolor{colColor1}{RGB}{255, 140, 115} % Coral  
%\definecolor{colColor2}{RGB}{255, 178, 140} % Soft Orange  
%\definecolor{colColor3}{RGB}{230, 153, 190} % Warm Pink  
%\definecolor{colColor4}{RGB}{200, 130, 220} % Light Orchid  
%\definecolor{colColor5}{RGB}{170, 110, 230} % Soft Violet  
%\definecolor{colColor6}{RGB}{140, 90, 240}  % Muted Purple  
%\definecolor{colColor7}{RGB}{120, 70, 250}  % Gentle Blue-Purple  
\definecolor{lastColColor}{RGB}{255, 255, 255}  % White
% Original Hierarchy colors
%\definecolor{AFragmentColor}{RGB}{255, 0, 0}  % Red
%\definecolor{BFragmentColor}{RGB}{0, 0, 255}  % Blue
%\definecolor{DFragmentColor}{RGB}{0, 203, 74}  % Green
%\definecolor{ASharedMemColor}{RGB}{200, 0, 0}  % Red
%\definecolor{BSharedMemColor}{RGB}{0, 0, 200}  % Blue
%\definecolor{WarpTileColor}{RGB}{0, 150, 74}  % Green
%\definecolor{BlockTileColor}{RGB}{0, 0, 0}  % Black
%\definecolor{GlobalPointsColor}{RGB}{255, 0, 0} % Red 
%\definecolor{GlobalQueriesColor}{RGB}{0, 0, 255}  % Blue
% New, less aggressive sequential pallettes
\definecolor{AFragmentColor}{HTML}{cb181d}
\definecolor{BFragmentColor}{HTML}{2171b5}
\definecolor{DFragmentColor}{HTML}{238b45}
\definecolor{ASharedMemColor}{HTML}{fb6a4a}
\definecolor{BSharedMemColor}{HTML}{6baed6}
\definecolor{GlobalPointsColor}{HTML}{fcae91}
\definecolor{GlobalQueriesColor}{HTML}{bdd7e7}
\definecolor{WarpTileColor}{HTML}{74c476}
\definecolor{BlockTileColor}{HTML}{bae4b3}

\newcommand{\getcolor}[1]{colColor#1}

\def\blockHeight{0.45}
\def\blockWidth{0.85}
\pgfmathsetmacro{\halfBlockHeight}{\blockHeight / 2}
\pgfmathsetmacro{\halfBlockWidth}{\blockWidth / 2}

\def\dimColors{colColor0, colColor1, colColor2, colColor3, colColor4, colColor5, colColor6, colColor7, lastColColor}

% Fragment drawings common parameters

% Base rectangle background
\newcommand\whiteSpace{0.2}

	% Compute width of a single register based on the fraction of a single chunk it takes up.
\pgfmathsetmacro\registerWidth{0.4}
\pgfmathsetmacro\registerHeight{\registerWidth / 2}
\pgfmathsetmacro\halfRegisterWidth{\registerWidth / 2}
\pgfmathsetmacro\halfRegisterHeight{\registerHeight / 2}

\newcommand\phasesPercent{0.8}
\pgfmathsetmacro\phaseWidth{\registerWidth * 4}
\pgfmathsetmacro\phaseHeight{\registerHeight * 8}

\pgfmathsetmacro\fragmentWidth{2 * \phaseWidth + (3 * \whiteSpace)}
\pgfmathsetmacro\fragmentHeight{2 * \phaseHeight + (3 * \whiteSpace)}

\tikzstyle fragment=[thick, draw=black, rounded corners= 5pt]

\newcommand\drawFragment[1]{
	\filldraw[style=fragment, fill=AFragmentColor] (0, 0) rectangle +(\fragmentWidth, -\fragmentHeight);

	\begin{scope}[xshift=\whiteSpace cm, yshift=-\whiteSpace cm]

	% Draw the four phases
		\foreach \row in {0,  1} {
			\foreach \col in {0,  1} {
				\pgfmathsetmacro\xoffset{\col * (\phaseWidth + \whiteSpace)}
				\pgfmathsetmacro\yoffset{-\row * (\phaseHeight + \whiteSpace)}
				\begin{scope}[xshift=\xoffset cm, yshift=\yoffset cm]
					\pgfmathsetmacro\phaseNum{int(\col * 2 + \row + 1)}
					\node[anchor=center, font=\tiny, text=white] at (\registerWidth * 2, 0.5 * \whiteSpace) {Phase \phaseNum};
					% Invoke the Phase drawing section now that we are in our local scope
					#1{\phaseNum}
				\end{scope}
			}
		}
	\end{scope}
}

\newcommand\getPhaseColor[1]{%
	\ifcase #1 \or colColor0%
		\or white%
		\or colColor1%
		\or white%
	\fi
}

% Bounding box around some of my diagrams
\newcommand\bbPad{0.2}
\pgfmathsetmacro\bottomPad{\bbPad + 0.2}
\newcommand\drawLabeledBoundingBox[3]{%
	\draw[#1] ($(current bounding box.north west) + (-\bbPad,\bbPad)$) rectangle ($(current bounding box.south east) + (\bbPad, -\bottomPad)$) node[anchor=south east, color=black, yshift=-0.125 cm, font=\small, text=#3] {#2};
}
