

% Register based fragment
\begin{tikzpicture}

	\newcommand\drawThreadPhase[1]{
		\edef\phaseNum{#1}
% Thread\register layout
		\foreach \phaseRow in {0, ..., 7} {
			\foreach \phaseCol in {0, ..., 3}{
				\coordinate (C) at (\phaseCol * \registerWidth, -\phaseRow * \registerHeight);
				\pgfmathsetmacro\threadIndex{int(\phaseRow * 4 + \phaseCol)}
				\def\chunkColor{\getPhaseColor{\phaseNum}}
				\filldraw[fill=\chunkColor, draw=black] (C) rectangle +(\registerWidth, -\registerHeight) node[midway, font=\tiny] {\texttt{T\threadIndex}};
                % Draw node labels for how A[n] matches up to this layout.
                \ifnum \threadIndex=0
                \coordinate (LabelPos) at ($(C)+(0.05, 0.1)$);
                \coordinate (ArrowEnd) at ($(C)+(\halfRegisterWidth + 0.1, 0)$);
                
                \pgfmathtruncatemacro\indexNum{\phaseNum - 1}
                % TODO would be nice to represent the registers somehow
                \node[font=\tiny, inner sep=0, text=white] (label) at (LabelPos) {\texttt{A[\indexNum]}};
                \draw[arrows= -{Classical TikZ Rightarrow[width=0.5mm, length=0.5mm]}, draw=white] (label.east) .. controls +(0.075, 0) .. (ArrowEnd);
                \fi
                
			}
		}
	}

	\drawFragment{ \drawThreadPhase }

\end{tikzpicture}
