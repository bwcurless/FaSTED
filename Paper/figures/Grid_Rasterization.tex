

% Block Tile Rasterization
\begin{tikzpicture}
	\newcommand\tileSpacing{0.25}
	\newcommand\tileDim{0.6}
    
	% Draw global memory layouts
	\newcommand\kHeight{1}
	\pgfmathsetmacro\mHeight{4 * \tileDim + 3 * \tileSpacing}
	\tikzstyle nodeStyle=[align=center, font=\footnotesize]
        \pgfmathsetmacro\queryXOffset{-\kHeight - \tileSpacing}
	\filldraw[xshift=\queryXOffset cm, yshift=0 cm, style=fragment, fill=GlobalPointsColor] (0, 0) rectangle +(\kHeight, -\mHeight) node[midway, style=nodeStyle] {Points\\ (DRAM)};
	\filldraw[xshift=0 cm, yshift=0.25 cm, style=fragment, fill=GlobalQueriesColor] (0, \kHeight) rectangle +(\mHeight, -\kHeight) node[midway, style=nodeStyle] {Query Points\\ (DRAM)};


	\pgfmathsetmacro\paddedTileDim{\tileDim + \tileSpacing}

		% Remember the last node that we need to draw an arrow from
	\def\innerTiles{2}
	\def\outerChunks{2}

		% Outer square rasterized chunks
	\pgfmathsetmacro\outerRange{int(\outerChunks - 1)}

	\foreach \outerRow in {0, ..., \outerRange} {
		\foreach \outerCol in {0, ..., \outerRange} {
			\pgfmathsetmacro\outerIndex{int(\outerRow * \outerChunks + \outerCol)}
			\begin{scope}[xshift=\outerCol * \innerTiles * \paddedTileDim cm,
				yshift=-\outerRow * \innerTiles * \paddedTileDim cm]

				% inner square rasterized chunks
				\pgfmathsetmacro\innerRange{int(\innerTiles - 1)}
				\foreach \row in {0, ..., \innerRange} {
					\foreach \col in {0, ..., \innerRange} {
						\pgfmathsetmacro\innerIndex{int(\row * \innerTiles + \col + 1)}
						\pgfmathsetmacro\tileIndex{int(\outerIndex * (\innerTiles * \innerTiles) + \innerIndex)}

						\coordinate (TileUL) at (\col * \paddedTileDim, -\row * \paddedTileDim);
						\filldraw[style=fragment, fill=BlockTileColor] (TileUL) rectangle ++(\tileDim, -\tileDim) node[midway, font=\tiny, align=center, color=black] {Block\\ Tile \tileIndex};

																		% Compute left and right nodes for arrows
						\coordinate (currentRightNode) at ($(TileUL) + (\tileDim, -\tileDim / 2)$);
						\coordinate (currentLeftNode) at ($(TileUL) + (0, -\tileDim / 2)$);
						\coordinate (currentBottomNode) at ($(TileUL) + (\tileDim / 2, -\tileDim)$);
						\coordinate (currentTopNode) at ($(TileUL) + (\tileDim / 2, 0)$);
						\tikzstyle arrow=[rounded corners=1pt, arrows=-{Latex[width=1.0mm, length=1.0mm]}]
																% Draw connecting arrows
						\newcommand\arrowOffset{\tileSpacing / 2}

						\newcommand\drawOuterShortArrow{
							\ifnum \innerIndex=1
								\draw[style=arrow] (lastRightNode) -- ++(\arrowOffset, 0) -- ++(0, \innerTiles * \paddedTileDim - \paddedTileDim) -- ++(\arrowOffset, 0);
							\fi
						}
						\newcommand\drawOuterLongArrow{
							\ifnum \innerIndex=1
								\draw[style=arrow] (lastBottomNode) -- ++(0 , -\tileSpacing / 2) -- ++(-4 * \paddedTileDim + \paddedTileDim, 0) -- ++(0, -\arrowOffset);
							\fi
						}
																		% Draw outer-raster-square connecting arrows
						\ifnum \outerIndex=1
							\drawOuterShortArrow{}
						\fi
						\ifnum \outerIndex=2
							\drawOuterLongArrow{}
						\fi
						\ifnum \outerIndex=3
							\drawOuterShortArrow{}
						\fi
																		% Draw normal inner connecting arrows
						\ifnum \col>0
							\draw[style=arrow] (lastRightNode) -- +(\tileSpacing, 0);
						\fi
																		% Draw last to first connecting arrows
						\ifnum \col=0
							\ifnum \row>0
								\draw[style=arrow] (lastBottomNode) -- ++(0, -\tileSpacing / 2) -- ++(-\innerTiles * \paddedTileDim + \paddedTileDim, 0) -- ++(0 , -\tileSpacing / 2);
							\fi
						\fi

																		% Save the last right node for the next iteration to connect
																		% arrows to it
						\coordinate (lastLeftNode) at (currentLeftNode);
						\coordinate (lastRightNode) at (currentRightNode);
						\coordinate (lastTopNode) at (currentTopNode);
						\coordinate (lastBottomNode) at (currentBottomNode);

					}
				}
			\end{scope}
		}
	}

\end{tikzpicture}
