\documentclass[tikz]{standalone}
\usepackage{tikz}
\usepackage{xcolor}
\usepackage{xparse}
\usetikzlibrary{calc}
\usetikzlibrary{matrix}

% Binary value zero padding taken from SO
\usepackage{xparse}
\ExplSyntaxOn
\int_new:N \g_fleet_number_of_zeros
\NewDocumentCommand\SetZeros{m}
  {
    \int_gset:Nn \g_fleet_number_of_zeros {#1}
  }
\NewDocumentCommand\PrependZeros{om}
  {
    \IfValueTF{#1}
      { \__fleet_count:ne {#1} {#2} }
      { \__fleet_count:ne {\g_fleet_number_of_zeros} {#2} }
  }
\cs_new:Npn \__fleet_count:nn #1#2
  {
    \exp_args:Nf \__fleet_prepend:nn
      { \int_max:nn { #1 - \str_count:n {#2} } { 0 } }
      {#2}
  }
\cs_generate_variant:Nn \__fleet_count:nn { ne }
\cs_new:Npn \__fleet_prepend:nn #1#2
  { \prg_replicate:nn {#1}{0} #2 }
\ExplSyntaxOff

% XOR routine taken from SO...
\ExplSyntaxOn
\NewExpandableDocumentCommand{\bitwiseXor}{mm}
 {
  \recuenco_bitwise_xor:nn { #1 } { #2 }
 }

\cs_new:Nn \recuenco_bitwise_xor:nn
 {
  \int_from_bin:e
   {
    \__recuenco_bitwise_xor:ee { \int_to_bin:n { #1 } } { \int_to_bin:n { #2 } }
   }
 }
\cs_generate_variant:Nn \int_from_bin:n { e }

\cs_new:Nn \__recuenco_bitwise_xor:nn
 {
  \__recuenco_bitwise_xor_binary:ee
   {
    \prg_replicate:nn
     {
      \int_max:nn { \tl_count:n { #1 } } { \tl_count:n { #2 } } - \tl_count:n { #1 }
     }
     { 0 }
     #1
   }
   {
    \prg_replicate:nn
     {
      \int_max:nn { \tl_count:n { #1 } } { \tl_count:n { #2 } } - \tl_count:n { #2 }
     }
     { 0 }
     #2
   }
 }
\cs_generate_variant:Nn \__recuenco_bitwise_xor:nn { ee }

\cs_new:Nn \__recuenco_bitwise_xor_binary:nn
 {
  \__recuenco_bitwise_xor_binary:w #1;#2;
 }
\cs_generate_variant:Nn \__recuenco_bitwise_xor_binary:nn { ee }

\cs_new:Npn \__recuenco_bitwise_xor_binary:w #1#2;#3#4;
 {
  \int_abs:n { #1-#3 }
  \tl_if_empty:nF { #2 } { \__recuenco_bitwise_xor_binary:w #2;#4; }
 }

\ExplSyntaxOff

\begin{document}

\definecolor{colColor0}{RGB}{255, 102, 102} % Soft Red  
\definecolor{colColor1}{RGB}{255, 140, 115} % Coral  
\definecolor{colColor2}{RGB}{255, 178, 140} % Soft Orange  
\definecolor{colColor3}{RGB}{230, 153, 190} % Warm Pink  
\definecolor{colColor4}{RGB}{200, 130, 220} % Light Orchid  
\definecolor{colColor5}{RGB}{170, 110, 230} % Soft Violet  
\definecolor{colColor6}{RGB}{140, 90, 240}  % Muted Purple  
\definecolor{colColor7}{RGB}{120, 70, 250}  % Gentle Blue-Purple  
\definecolor{lastColColor}{RGB}{255, 255, 255}  % White
\definecolor{AFragmentColor}{RGB}{255, 0, 0}  % Red
\definecolor{BFragmentColor}{RGB}{0, 0, 255}  % Blue
\definecolor{DFragmentColor}{RGB}{0, 255, 0}  % Green

\newcommand{\getcolor}[1]{colColor#1}


\def\blockHeight{1}
\def\blockWidth{2}
\pgfmathsetmacro{\halfBlockHeight}{\blockHeight / 2}
\pgfmathsetmacro{\halfBlockWidth}{\blockWidth / 2}

\def\dimColors{colColor0, colColor1, colColor2, colColor3, colColor4, colColor5, colColor6, colColor7, lastColColor}
	
% Global Memory layout
\begin{tikzpicture}[scale=1]
	\foreach \color [count=\col from 0] in \dimColors {
		\foreach \row in {0,..., 7}{

			% Rectangle coordinates
			\pgfmathsetmacro{\yPos}{-\row * \blockHeight}
			\pgfmathsetmacro{\xPos}{\col * \blockWidth}
			\coordinate (UL) at (\xPos, \yPos);
			\coordinate (BR) at  (\xPos + \blockWidth, \yPos - \blockHeight);
			% Center of rectangle
			\coordinate (C) at ($ (UL)!.5!(BR) $);

			% Main rectangle
			\fill[color=\color, draw=black] (UL) rectangle (BR);

			% Label rows with Point number
			\ifnum\col=0
				\pgfmathsetmacro{\pointNum}{int(\row + 1)}
				\pgfmathsetmacro{\poffset}{-(\halfBlockWidth + 0.5)}
				\node[font=\large, xshift=\poffset cm] at (C) {$P_\pointNum$};
			\fi


			% Add dimension labels
			\pgfmathsetmacro{\baseDim}{int((\col) * 8 + 1)}
			% Handle last col differently since it goes to d
			\ifnum\col=8
				\node[font=\large] at (C)  {\baseDim ...$d$};
			\else
				% Label with thread index
				\pgfmathsetmacro{\cellIndex}{int((\row) * 8 + \col)}
				\node[anchor=north west, font=\tiny] at (UL) {T\cellIndex};
				\pgfmathsetmacro{\maxDim}{int(\baseDim + 7)}
				\node[font=\large] at (C)  {\baseDim -\maxDim};
			\fi

	}};
\end{tikzpicture}

% Shared Memory layout
\begin{tikzpicture}[scale=1]
	\node[font=\large] at (8, 0) {\underline{Shared Memory Banks}};
	\begin{scope}[shift={(0,-1.25)}]
		
	\foreach \col in {0,...,7} {
		\foreach \row in {0,..., 7}{

			% Base calculations
			\pgfmathsetmacro{\yPos}{-\row * \blockHeight}
			\pgfmathsetmacro{\xPos}{\col * \blockWidth}
			\coordinate (UL) at (\xPos, \yPos);
			\coordinate (BR) at  (\xPos + \blockWidth, \yPos - \blockHeight);
			% Center of rectangle
			\coordinate (C) at ($ (UL)!.5!(BR) $);

			% Compute swizzled column
			\pgfmathsetmacro{\swizzledCol}{int(\bitwiseXor{\col}{\row})}

			% Draw rect with appropriate column color
			\def\cellColor{\getcolor{\swizzledCol}}
			\fill[color=\cellColor, draw=black] (UL) rectangle (BR);

			% Label rows with Point number
			\ifnum\col=0
				\pgfmathsetmacro{\pointNum}{int(\row + 1)}
				\pgfmathsetmacro{\poffset}{-(\halfBlockWidth + 0.5)}
				\node[font=\large, xshift=\poffset cm] at (C) {$P_\pointNum$};
			\fi

			\pgfmathsetmacro{\headerOffset}{(\halfBlockHeight + 0.5)}
			% Label columns with bank number
			\ifnum\row=0
			% Add column number in binary
				\pgfmathsetmacro{\binaryColumn}{int(bin(\col)}
				\newcommand\paddedBinaryAddress{\PrependZeros[3]{\binaryColumn}}

				\pgfmathsetmacro{\bank}{int(\col+1)}
				\node[font=\large, yshift=\headerOffset cm] at (C) {$ b_\mathtt{\paddedBinaryAddress} $};
			\fi

			\pgfmathsetmacro{\binaryRow}{int(bin(\row))}
			% Add XOR values to right side of figure
			\ifnum\col=7
				\pgfmathsetmacro{\xorOffset}{(\halfBlockWidth + 0.1)}
				% Add header for XOR column...
				\ifnum\row=0
					\node[anchor=west, font=\small, xshift=\xorOffset cm, yshift=\headerOffset cm] at (C) {\underline{XOR}};
				\fi
				\newcommand\paddedBinaryRow{\PrependZeros[3]{\binaryRow}}
				\node[anchor=west, font=\small, xshift=\xorOffset cm] at (C) {\texttt{0b\paddedBinaryRow}};
			\fi

			% Add dimension range for each block
			\pgfmathsetmacro{\baseDim}{int(8 * \swizzledCol + 1)}
			\pgfmathsetmacro{\maxDim}{int(\baseDim + 7)}
			\node[font=\large] at (C)  {\baseDim -\maxDim};


}};
	\end{scope}
\end{tikzpicture}

% Fragment drawings common parameters

% Base rectangle background
\newcommand\whiteSpace{0.5}

	% Compute width of a single register based on the fraction of a single chunk it takes up.
\pgfmathsetmacro\registerWidth{0.75}
\pgfmathsetmacro\registerHeight{0.375}
\pgfmathsetmacro\halfRegisterWidth{\registerWidth / 2}
\pgfmathsetmacro\halfRegisterHeight{\registerHeight / 2}

\newcommand\phasesPercent{0.8}
\pgfmathsetmacro\phaseWidth{\registerWidth * 4}
\pgfmathsetmacro\phaseHeight{\registerHeight * 8}

\pgfmathsetmacro\fragmentWidth{2 * \phaseWidth + (3 * \whiteSpace)}
\pgfmathsetmacro\fragmentHeight{2 * \phaseHeight + (3 * \whiteSpace)}

\tikzstyle fragment=[ultra thick, rounded corners= 5pt]

\newcommand\drawFragment[1]{
	\draw[style=fragment, draw=AFragmentColor] (0, 0) rectangle +(\fragmentWidth, -\fragmentHeight);

	\begin{scope}[xshift=\whiteSpace cm, yshift=-\whiteSpace cm]

	% Draw the four phases
		\foreach \row in {0,  1} {
			\foreach \col in {0,  1} {
				\pgfmathsetmacro\xoffset{\col * (\phaseWidth + \whiteSpace)}
				\pgfmathsetmacro\yoffset{-\row * (\phaseHeight + \whiteSpace)}
				\begin{scope}[xshift=\xoffset cm, yshift=\yoffset cm]
					\pgfmathsetmacro\phaseNum{int(\col * 2 + \row + 1)}
					\node[anchor=south] at (\registerWidth * 2, 0) {Phase \phaseNum};
					% Invoke the Phase drawing section now that we are in our local scope
					#1{\phaseNum}
				\end{scope}
			}
		}
	\end{scope}
}

\newcommand\getPhaseColor[1]{%
	\ifcase #1 \or colColor0%
		\or white%
		\or colColor1%
		\or white%
	\fi
}

% Chunk based fragment
\begin{tikzpicture}

	\newcommand\drawChunkPhase[1]{
			% edef worked here because I was defining \def\phaseNum{\phaseNum} which recurses
			% infinitely. \edef worked because it expanded \phaseNum before assigning it to 
			% the new macro.
		\def\phaseNumIn{#1}
% Thread\register layout
		\foreach \row in {0, ..., 7} {
			\coordinate (C) at (0, -\row * \registerHeight);
			\ifnum \phaseNumIn<3
				\def\firstDim{1}
			\else
				\def\firstDim{9}
			\fi
			\pgfmathsetmacro\maxDim{int(\firstDim + 7)}

			\def\chunkColor{\getPhaseColor{\phaseNumIn}}
			\pgfmathsetmacro\thread{int((\phaseNumIn - 1) * 8 + \row)}
			\filldraw[fill=\chunkColor, draw=black] (C) node[anchor=north west, font=\tiny] {\texttt{T\thread}} rectangle +(4 * \registerWidth, -\registerHeight) node[midway, font=\small] {\firstDim-\maxDim} ;
		}
	}

	\drawFragment{\drawChunkPhase}

\end{tikzpicture}

% Register based fragment
\begin{tikzpicture}

	\newcommand\drawThreadPhase[1]{
		\edef\phaseNum{#1}
% Thread\register layout
		\foreach \phaseRow in {0, ..., 7} {
			\foreach \phaseCol in {0, ..., 3}{
				\coordinate (C) at (\phaseCol * \registerWidth, -\phaseRow * \registerHeight);
				\pgfmathsetmacro\threadIndex{int(\phaseRow * 4 + \phaseCol)}
				\def\chunkColor{\getPhaseColor{\phaseNum}}
				\filldraw[fill=\chunkColor, draw=black] (C) rectangle +(\registerWidth, -\registerHeight) node[midway, font=\small] {\texttt{T\threadIndex}};
			}
		}
	}

	\drawFragment{ \drawThreadPhase }

\end{tikzpicture}

% Basic TC MMA
\begin{tikzpicture}
		\newcommand\fm{4.5}
		\newcommand\fn{3}
		\newcommand\fk{4.5}

		% Draw B
		\pgfmathsetmacro\bdXShift{1.5}
		\draw[style=fragment,draw=BFragmentColor, ultra thick] (0, 0) rectangle +(\fn, -\fk);
		\matrix (m) [xshift=\bdXShift cm, yshift=-2.25cm, 
		matrix of math nodes,
		row sep=0.5em, column sep=0em,
		nodes={anchor=center}] {
				C_{1,1} & \cdots & C_{8,1} \\
				C_{1,2} & \cdots & C_{8,2} \\
				\vdots & \ddots & \vdots \\
				C_{1,15} & \cdots & C_{8,15} \\
				C_{1,16} & \cdots & C_{8,16} \\
};

		% Draw D
		\pgfmathsetmacro\dXShift{0}
		\pgfmathsetmacro\dYShift{-\whiteSpace - \fk}
		\draw[style=fragment,draw=DFragmentColor, xshift=\dXShift cm, yshift=\dYShift cm, ultra thick] (0, 0) rectangle +(\fn, -\fm);
		\pgfmathsetmacro\adYShift{-7.25}
		\matrix (m) [xshift=\bdXShift cm, yshift=\adYShift cm, 
		matrix of math nodes,
		row sep=0.5em, column sep=0em,
		%left delimiter={[}, right delimiter={]},
		nodes={anchor=center}] {
				D_{1,1} & \cdots & D_{1,8} \\
				D_{2,1} & \cdots & D_{2,8} \\
				\vdots & \ddots & \vdots \\
				D_{15,1} & \cdots & D_{15,8} \\
				D_{16,1} & \cdots & D_{16,8} \\
};

		% Draw A
		\pgfmathsetmacro\aXShift{-\whiteSpace - \fk}
		\pgfmathsetmacro\aYShift{-\whiteSpace - \fm}
		\draw[style=fragment,draw=AFragmentColor, xshift=\aXShift cm, yshift=\aYShift cm, ultra thick] (0, 0) rectangle +(\fk, -\fm);
		\matrix (m) [xshift=-2.75 cm, yshift=\adYShift cm, 
		matrix of math nodes,
		row sep=0.5em, column sep=0.5em,
		nodes={anchor=center}] {
				Q_{1,1} & Q_{1,2} & \cdots & Q_{1,16} \\
				Q_{2,1} & Q_{2,2} & \cdots & Q_{2,16} \\
				\vdots & \vdots & \ddots & \vdots \\
				Q_{15,1} & Q_{15,2} & \cdots & Q_{15,16} \\
				Q_{16,1} & Q_{16,2} & \cdots & Q_{16,16} \\
};
		
\end{tikzpicture}

% Warp MMA
\begin{tikzpicture}
		\newcommand\fm{2}
		\newcommand\fn{1}
		\newcommand\fk{2}
		
\end{tikzpicture}

\end{document}
